\documentclass[a4paper,10pt]{article}
\usepackage[utf8]{inputenc}
\usepackage[normalem]{ulem}
%opening
\title{}
\author{}

\begin{document}

\maketitle
\section{Modelo Relacional}\label{modelo-relacional}

\subsection{Entidades}\label{entidades}

Civil(\uline{dni}, nombre, apellido)\\
Superheroe(\uline{shid}, nombre, color de disfraz, \dashuline{dni})\\
Habilidad(\uline{habilidadId}, nombre)\\
OrganizacionDelictiva(\uline{mafiaId}, nombre)\\
Direccion(\uline{idDireccion}, calle, altura, \dashuline{idBarrio})\\
Barrio(\uline{idBarrio}, nombre)\\
TipoDeRelacion(\uline{idTipoDeRelacion}, nombre)\\
Incidente(\uline{idIncidente}, fecha, calle\_1, calle\_2,
\dashuline{idTipoIncidente})\\
TipoIncidente(\uline{idTipoIncidente}, nombre)\\
Seguimiento(\uline{numero}, fecha, descripción, conclusión,
estado,\uuline{idIncidente}, \dashuline{placa})\\
Departamento(\uline{idDepartamento}, nombre, descripción)\\
Oficial(\uline{placa},dni, nombre, apellido, rango, fechaIngreso, tipo,
\dashuline{idDepartamento}) \\
Investigador(\uuline{placa})\\
RolOficial(\uline{idResponsabilidad}, descripción)\\
RolCivil(\uline{idRolCivil})\\
Cargo(\uline{idCargo}, fechaInicio, fechaFin, \uline{idDesignación})\\
Sumario(\uline{idSumario}, fecha, observación, resultado, \dashuline{placa},
\dashuline{idCargo})\\
Designacion(\uline{idDesignación}, nombre) \\

\subsection{Relaciones}\label{relaciones}

Tiene(\uuline{shid}, \uuline{habilidadId})\\
ArchienemigoDe(\uuline{shid},\uuline{dni})\\
EsContactadoPor(\uuline{shid},\uuline{dni})\\
EstaCompuestaPor(\uuline{mafiaId}, \uuline{dni})\\
Conocimiento(\uuline{conocedor},\uuline{conocido},
fechaConocimiento, \dashuline{idTipoDeRelación})

\subsection{Consideraciones adicionales}\label{consideraciones-adicionales}

Como en la relacion \emph{identificadoComo} ambas entidades participan
parcialmente, decidimos poner la foreign key en superheroe, asumiendo
que esto nos generaría menos elementos nulos en la base de datos.

\end{document}
