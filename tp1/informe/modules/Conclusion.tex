\section{Conclusión}

Se logró diseñar un modelo del problema que cumple con el enunciado.  
Este modelo facilita la respuesta de las consultas requeridas. 
La implementación física en un motor rdbms resultó casi en una conversion 1 a 1 con respecto al modelo relacional. Se utilizaron las funcionalidades de triggers para garantizar el cumplimiento de restricciones que surgieron del análisis del problema a resolver.
Una de las tareas que facilita todo el proceso de comprensión del problema y su posterior implementación en una base de datos es claramente el diseño de un modelo entidad relación previo, ya que luego puede pasarse, mediante reglas bien establecidas, a un modelo relacional y permite establecer el conjunto de asunciones y restricciones que deben tomarse sin ambigüedades.