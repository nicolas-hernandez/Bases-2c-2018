\section{Desarrollo}

\subsection{DER} 

\subsection{Modelo Relacional}\label{modelo-relacional}

\subsubsection{Entidades}\label{entidades}

Civil(\uline{dni}, nombre, apellido)\\
%PK = \{dni\}\\
%CK = \{dni\}\\

Superheroe(\uline{idSh}, nombre, color de disfraz, \dashuline{dni})\\
%PK = \{idSH\}\\
%CK = \{idSh\}\\
%FK = \{dni\}\\

Habilidad(\uline{IdHabilidad}, nombre)\\

OrganizacionDelictiva(\uline{idMafia}, nombre)\\

Direccion(\uline{idDireccion}, calle, altura, \dashuline{idBarrio})\\

Barrio(\uline{idBarrio}, nombre)\\

TipoDeRelacion(\uline{idTipoDeRelacion}, nombre)\\

Incidente(\uline{idIncidente}, fecha, calle\_1, calle\_2,
\dashuline{idTipoIncidente}, \dashuline{idDireccion})\\

TipoIncidente(\uline{idTipoIncidente}, nombre)\\

Seguimiento(\uline{numero}, fecha, descripción, conclusión,
estado,\uuline{idIncidente}, \dashuline{placa}, \dashuline{idEstadoSeg})\\

EstadoSeguimiento(\uline{idEstadoSeg}, estado) \\

Departamento(\uline{idDepartamento}, nombre, descripción)\\

Oficial(\uline{placa},dni, nombre, apellido, rango, fechaIngreso, tipo,
\dashuline{idDepartamento}) \\

Investigador(\uuline{placa})\\

RolOficial(\uline{idResponsabilidad}, descripción)\\

RolCivil(\uline{idRolCivil}, nombre)\\

Asignación(\uline{idAsignación}, fechaInicio, \dashuline{idDesignación}, \dashuline{placa})\\

Sumario(\uline{idSumario}, fecha, observación, resultado, \dashuline{placa},
\dashuline{idAsignación}, \dashuline{idEstadoSum})\\

EstadoSumario(\uline{idEstadoSum}, estado) \\

Designacion(\uline{idDesignación}, nombre) \\

\subsubsection{Relaciones}\label{relaciones}

OficialSeInvolucró(\uuline{placa}, \uuline{idIncidente}, \uuline{idResponsabilidad})\\

Posee(\uuline{idSh}, \uuline{idHabilidad})\\

ArchienemigoDe(\uuline{idSh},\uuline{dni})\\

EsContactadoPor(\uuline{idSh},\uuline{dni})\\

SuperParticipó(\uuline{idSh}, \uuline{idIncidente})\\

EstaCompuestaPor(\uuline{idMafia}, \uuline{dni})\\

Conocimiento(\uuline{conocedor},\uuline{conocido},
fechaConocimiento, \dashuline{idTipoDeRelación})\\

seInvolucraron(\uuline{dni}, \uuline{idIncidente}, \dashuline{idRolCivil})\\

ViveEn(\uuline{dni}, \uuline{idDirección}, fechaInicio)\\

\subsubsection{Consideraciones adicionales}\label{consideraciones-adicionales}

Como en la relacion \emph{identificadoComo} ambas entidades participan
parcialmente, decidimos poner la foreign key en superheroe, asumiendo
que esto nos generaría menos elementos nulos en la base de datos.

\subsubsection{Restricciones del modelo}

\begin{itemize}

\item\uline{\textbf{Asignación}}:
\begin{itemize}
\item fecha de inicio mayor o igual a fecha de ingreso del oficial
\item Asumimos que solo puede tener una asignación a una designación al mismo tiempo. Cuando termina una
empieza otra asignación en otra.
\end{itemize}


\item\uline{\textbf{sumario:}}
\begin{itemize}
\item fecha del sumario mayor a fecha de ingreso del oficial a investigar.
\item Investigador del sumario no puede investigarse a si mismo.
\item fecha del sumario mayor o igual a la fecha de inicio de la asignación.
\item si concluyó entonces tiene un resultado. 
\end{itemize}
\item\uline{\textbf{oficial:}}
\begin{itemize}
\item dni unicos
\item  si se involucra en un incidente, la fecha de este último tiene que ser mayor a la fecha
de ingreso del oficial.
\end{itemize}
\item\uline{\textbf{seguimiento:}}
\begin{itemize}
\item Solo puede ser seguida si su estado es " en proceso"
\item Si es seguida, lo es por un oficial cuya fecha de ingreso sea menor a la fecha del seguimiento
\item Fecha del seguimiento mayor a fecha del incidente del relacionado.
\item Una vez que un proceso esta en estado "cerrado" no puede cambiar de estado y deja de estar seguido por un oficial.
\item Conclusion solo tiene un valor si el estado del seguimiento es cerrado.
\end{itemize}
\item\uline{\textbf{roloficial:}}
\begin{itemize}
\item Asumimos que un oficial en un incidente puede cumplir muchos roles. Por eso es p su cardinalidad en la ternaria entre esta, oficial e incidente por la interrelacion "OficialseInvolucro".
\end{itemize}

\item\uline{\textbf{Super héroe:}}
\begin{itemize}
\item No puede ser archienemigo de la misma persona de la que es identificado como.
\item No puede ser identificado como una persona que pertence a una organizacion delictiva.
\item Asumimos que no puede formar parte de una organización delictiva.
\item No puede participar como superhéroe en el incidente al mismo tiempo que su correspondiente 
civil (si es que se conoce su identidad) se involucra como civil.
\end{itemize}
\item\uline{\textbf{Civil/Persona:}}
\begin{itemize}
\item  Elejimos llamar Civil a tal entidad porque entendemos que los oficiales, aunque a veces no lo parezcan, son personas. En cualquier caso creemos que esta entidad representa el aspecto cívico de un habitante de ciudad Gótica con sus propias interrelaciones y atributos.
En este modelo asumimos que no va a existir un oficial que sea civil.
\item En este modelo simplificamos para que un civil no pueda volver a vivir en una dirección.

\end{itemize}
\end{itemize}

\subsection{Modelo Físico}