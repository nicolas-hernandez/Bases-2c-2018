\section{Introducción}

En este trabajo práctico se captura la realidad de la ciudad gótica definida por enunciado y se la implementa en una base de datos relacional. Para esto primero se diseñó el MER (Modelo Entidad Relación) y el MR (Modelo Relacional) y se definen las restricciones y asunciones a tener en cuenta. En base a lo mencionado se procede implementar la base de datos en Postgres SQL. Las restricciones serán implementadas mediante function triggers.

Como parte del enunciado, también, se pide responder a una serie de consultas, las cuales influyen intrinsecamente en el diseño realizado, ya que debemos poder responder a las mismas. Las consultas se listan a continuación:

\begin{itemize}
\item Listado de incidentes en un rango de fechas, mostrando los datos de las per-
sonas y policías involucrados con el rol que jugó cada uno en el incidente
\item  Dada una organización delictiva, el detalle de incidentes en que participaron
las personas que componen dicha organización
\item  La lista de todos los oficiales con sus rangos, de un departamento dado.
\item  El ranking de oficiales que participaron en más incidentes
\item  Los barrios con mayor cantidad de incidentes.
\item  Todos los oficiales sumariados que participaron de algún incidente.
\item  Las personas involucradas en incidentes ocurridos en el barrio donde viven
\item  Los superheroes que tienen una habilidad determinada
\item  Los superheroes que han participado en algún incidente.
\item  Listado de todos los incidentes en donde estuvieron involucrados superheroes
y fueron causados por los "archienemigos" de los superheroes involucrados.
\end{itemize}